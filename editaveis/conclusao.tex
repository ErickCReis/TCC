\chapter[Conclusão]{Conclusão}

Durante a elaboração deste trabalho foi possível destacar as dificuldades encontradas por alunos e desenvolvedores durante a construção de um projeto, seja ele um projeto pessoal ou algo desenvolvido para um disciplina da graduação por exemplo.

Como primeira etapa, este trabalho visou dar uma introdução ao funcionamento dos microcontroladores, mas principalmente mostrar como é o mercado brasileiro de módulos e componentes utilizados em prototipagem, mostrando como as shield podem ajudar no processo como o todo.

Desenvolver shields voltadas para microcontroladores se mostra uma ótima forma de ajudar o desenvolvimento de projetos e consequentemente o aprendizado na área. Utilizando como base a Launchpad MSP430F5529 este projeto pode se tornar extremamente atrativo para os alunos de Engenharia Eletrônica da UnB.

Para finalizar este projetos ainda existem algumas etapas importantes que ainda não foram totalmente exploradas neste documento, principalmente quando se trata da integração dos componentes tanto em hardware quanto em software. Além disso este projeto exige um certo período de testes e algumas validações para realmente torná-lo viável.