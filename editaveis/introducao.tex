\chapter[Introdução]{Introdução}

O mercado de microcontroladores tem tido recorde de vendas entre os anos 2010 e 2018 \cite{Lineback2019}, não se tratando apenas de número de vendas é nítido o quão acessível produtos eletrônicos tem se tornado para o dia a dia das pessoas.

Entender os princípios de funcionamento de um microcontrolador, bem como utizá-lo em algum projeto, é uma algo básico e necessário para que um aluno de graduação saia minimamente preparado para esse mercado que ainda tem perspectivas de crescimento.

Na Universidade de Brasília, Faculdade Gama, o contato efetivo com microntroladores acontece apenas no 6º semestre, e como esperado, os alunos têm que elaborar um projeto final no qual é necessário utilizar alguns periféricos, e é nesse ponto que as dificuldades começam a aparecer. Elaborar um projeto que atenda as expectativas da disciplina normalmente evolve a união de alguns módulos e circuitos externos que, muitas vezes, apresentam problemas que envolvem duas áreas: dificuldade de entendimento do funcionamento e operação dos módulos e, principalmente, empecilhos relacionados à conexão e mal contatos com jumpers.

Visando reduzir o tempo gasto resolvendo esses problemas esse trabalho propõe a elaboração de um kit de desenvolvimento modular utilizando a família de microcontroladores MSP430, da Texas Instruments.

\section{Justificativa}

A variedades de módulos e tipos de conexão dificultam o uso de funcionalidade básicas dos microcontroladores, além disso outro ponto que atrapalha não só o uso, mas também o desenvolvimento de protótipo é a falha nas conexões, jumpers e fios, que por sua vez são extremamente suscetíveis à mal contato. A falta de módulos integrados, conhecidos como shields ou boosterpacks, para a plataforma MSP430 torna esse desafio ainda maior.

\section{Importância}

O uso de microcontroladores no mercado é muito amplo, e o aprendizado nessa área não é simples e tem uma longa cadeia de disciplinas, entretanto é muito claro que somente trabalhando em projetos é possível entender as particularidades do seu uso. Por isso uma plataforma que facilite o desenvolvimento é extremamente importante, já que com uma estrutura robusta é possível focar apenas no projeto e não em erros provenientes da montagem e/ou conexões.

\section{Objetivos}

Desenvolver um sistema modular de baixo custo para o ensino de eletrônica e sistemas embarcados, que possua diversas funcionalidades de sensoriamento, acionamento e comunicação.

\subsection{Objetivos Específicos}

\begin{itemize}
	\item Escolher o microcontrolador e o hardware que servirá de base para os módulos a serem elaborados.
	\item Listar e definir interfaces, sensores e atuadores que farão parte dos módulos.
	\item Construir os módulos compatíveis com a plataforma e que sejam economicamente acessíveis para alunos.
	\item Disponibilizar a documentação e o firmware necessário para a comunicação e interação com os módulos.
\end{itemize}

\section{Metodologia}

A realização desse trabalho se dará primeiramente entendendo as necessidades que o produto em si deve cumprir, ou seja, quais funções devem ser desempenhadas, este ponto será feito através de pesquisas com o objetivo de identificar funcionalidades mais comuns e que explorem uma gama de características de um microcontrolador, além disso professores que ministram disciplinas da área serão consultados.

Posteriormente será feita uma pesquisa de componentes que cumpram as funcionalidades especificadas, visando também encontrar opções disponíveis no mercado brasileiro, em seguida será planejado a organização dos módulos nas shields buscando trazer a melhor usabilidade.

Por fim um conjunto de códigos serão desenvolvidos para servir como exemplo de utilização das shields desenvolvidas, esta etapa é importante para facilitar a utilização do produto. Além disso também será desenvolvido material de apoio explicando cada uma das shields e como deve ser sua utilização.

\section{Organização}

Este documento está organizado da seguinte maneira:

\begin{itemize}
	\item Referencial teórico com uma explicação do funcionamento dos microcontroladores de forma geral, projetos com escopo semelhante e, além disso, uma pesquisa sobre produtos disponíveis no mercado brasileiro que podem contribuir no desenvolvimento do trabalho;
	\item Desenvolvimento do projeto em si, justificando todas as escolhas e detalhando a organização dos shields em termos de esquemático e layout, bem como ferramentas a serem utilizadas e um cronograma geral das próximas etapas.
	\item Conclusão com observações importantes desse trabalho.
\end{itemize}